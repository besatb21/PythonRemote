\documentclass[a4paper,12pt]{article}
\setlength{\textheight}{10in}\setlength{\textwidth}{6.5in}\setlength{\topmargin}{-0.125in}\setlength{\oddsidemargin}{-.2in}\setlength{\evensidemargin}{-.2in}\setlength{\headsep}{0.2in}\setlength{\footskip}{0pt}
\usepackage[backend=biber,style=ieee]{biblatex}
\renewcommand*{\bibfont}{\footnotesize}
%\addbibresource{ref.bib}
\usepackage{amsmath}\usepackage{fancyhdr}\usepackage{enumitem}\usepackage{hyperref}\usepackage{graphicx}\usepackage{subcaption}\usepackage{libertine}\usepackage{bm}\usepackage{amssymb} \usepackage{tikz}\usepackage{fontawesome}
\usepackage{listings}
\pagestyle{fancy}\lhead{Name: Besjana Jaçaj }\fancyfoot{}
\begin{document}
        \textbf{Iterators}\\
        As we already know, Python has several objects that implement an iterative protocol, namely:
        lists, strings,dictionaries etc.
        However, there are cases where working with such iterable objects is costly memory-wise.
        In this case, we can use \textbf{Iterator}.
        Example:\\
        \item \begin{lstlisting}[language=Python]
        from math import sqrt


def is_prime(n):
    for i in range(2, int(sqrt(n)) + 1):
        if n % i == 0:
            return False
    return True


def get_n_primes(n):
    primes = []
    i = 2
    while len(primes) != n:
        if is_prime(i):
            primes.append(i)
        i += 1
    return primes


lst = get_n_primes(1000000)
for elem in lst:
    print(elem)

\end{lstlisting}

    \printbibliography
\end{document}




